%% LyX 2.0.3 created this file.  For more info, see http://www.lyx.org/.
%% Do not edit unless you really know what you are doing.
\documentclass[twoside,english]{paper}
\usepackage{lmodern}
\renewcommand{\ttdefault}{lmodern}
\usepackage[T1]{fontenc}
\usepackage[latin9]{inputenc}
\usepackage[a4paper]{geometry}
\geometry{verbose,tmargin=3cm,bmargin=2.5cm,lmargin=2cm,rmargin=2cm}
\usepackage{color}
\usepackage{babel}
\usepackage{float}
\usepackage{bm}
\usepackage{amsthm}
\usepackage{amsmath}
\usepackage{amssymb}
\usepackage{graphicx}
\usepackage{esint}
\usepackage[unicode=true,pdfusetitle,
 bookmarks=true,bookmarksnumbered=false,bookmarksopen=false,
 breaklinks=false,pdfborder={0 0 0},backref=false,colorlinks=false]
 {hyperref}
\usepackage{breakurl}
\usepackage{makeidx}

\makeatletter

%%%%%%%%%%%%%%%%%%%%%%%%%%%%%% LyX specific LaTeX commands.
%% Because html converters don't know tabularnewline
\providecommand{\tabularnewline}{\\}

%%%%%%%%%%%%%%%%%%%%%%%%%%%%%% Textclass specific LaTeX commands.
\numberwithin{equation}{section}
\numberwithin{figure}{section}

%%%%%%%%%%%%%%%%%%%%%%%%%%%%%% User specified LaTeX commands.
\usepackage{babel}

\@ifundefined{showcaptionsetup}{}{%
 \PassOptionsToPackage{caption=false}{subfig}}
\usepackage{subfig}
\makeatother

\usepackage{listings}


\begin{document}

\title{Generalised parton distributions}

\author{Valerio Bertone}

\tableofcontents{}

In this set of notes I collect the technical aspects concerning
generalised parton distributions (GPDs). Since the computation GPDs
introduces new kinds of convolution integrals, a strategy aimed at
optimising the numerics needs to be devised.

\section{Evolution equations}

We start by discussing the implementation of the evolution
equations. I will try to stick to Ref.~\cite{Diehl:2003ny} for the
general notation. First, I introduce the main variables involved in
the computation of GPDs:
\begin{itemize}
\item $x\equiv\frac{Q^2}{2pq}$ is the usual Bjorken x defined in the
  range $[-1:1]$,
\item $\xi= \frac{p^+-p'^+}{p^++p'^+}$:
\item $t = (p-p')^2$ is the usual $t$-channel squared energy,
\item $\mu$ the renormalisation scale
\end{itemize}
Sect.~3.1 of Ref.~\cite{Diehl:2003ny} gives a comprehensive review of
the relevant variables and notation.



\newpage

\begin{thebibliography}{alp}

%\cite{Diehl:2003ny}
\bibitem{Diehl:2003ny}
  M.~Diehl,
  %``Generalized parton distributions,''
  Phys.\ Rept.\  {\bf 388} (2003) 41
  doi:10.1016/j.physrep.2003.08.002, 10.3204/DESY-THESIS-2003-018
  [hep-ph/0307382].
  %%CITATION = doi:10.1016/j.physrep.2003.08.002, 10.3204/DESY-THESIS-2003-018;%%
  %1016 citations counted in INSPIRE as of 30 Oct 2019

\end{thebibliography}




\end{document}
