\documentclass[10pt,a4paper]{article}
\usepackage{amsmath,amssymb,bm,makeidx,subfigure}
\usepackage[italian,english]{babel}
\usepackage[center,small]{caption}[2007/01/07]
\usepackage{fancyhdr}
\usepackage{color}

\definecolor{blu}{rgb}{0,0,1}
\definecolor{verde}{rgb}{0,1,0}
\definecolor{rosso}{rgb}{1,0,0}
\definecolor{viola}{rgb}{1,0,1}
\definecolor{arancio}{rgb}{1,0.5,0}
\definecolor{celeste}{rgb}{0,1,1}
\definecolor{rosa}{rgb}{1,0.3,0.5}

\oddsidemargin = 12pt
\topmargin = 0pt
\textwidth = 440pt
\textheight = 650pt

\makeindex

\begin{document}

\section{Structure of the observables}

Let us start from Eq.~(2.6) of Ref.~\cite{Scimemi:2017etj}, that is
the fully differential cross section for lepton-pair production in the
region in which the TMD factorisation applies, $i.e.$ $q_T \ll
Q$. After some minor manipulations, it reads:
\begin{equation}\label{eq:crosssection}
  \frac{d\sigma}{dQ dy dq_T} =
  \frac{16\pi\alpha^2q_T}{9 Q^3} H(Q,\mu) \sum_q C_q(Q)
  \int\frac{d^2\mathbf{b}}{4\pi} e^{i \mathbf{b}\cdot \mathbf{q}_T} \overline{F}_q(x_1,\mathbf{b};\mu,\zeta) \overline{F}_{\bar{q}}(x_2,\mathbf{b};\mu,\zeta)\,,
\end{equation}
where $Q$, $y$, and $q_T$ are the invariant mass, the rapidity, and
the transverse momentum of the lepton pair, respectively, while
$\alpha$ is the electromagnetic coupling, $H$ is the appropriate QCD
hard factor that can be perturbatively computed, and $C_q$ are the
effective electroweak charges. In addition, the variables $x_1$ and
$x_2$ are functions of $Q$ and $y$ and are given by:
\begin{equation}\label{eq:Bjorkenx12}
  x_{1,2} = \frac{Q}{\sqrt{s}}e^{\pm y}\,,
\end{equation}
being $\sqrt{s}$ the centre-of-mass energy of the collision. In
Eq.~(\ref{eq:crosssection}) we are using the short-hand notation:
\begin{equation}
\overline{F}_q(x,\mathbf{b};\mu,\zeta) \equiv xF_q(x,\mathbf{b};\mu,\zeta)\,,
\end{equation}
that is convenient for the implementation. The scales $\mu$ and
$\zeta$ are introduced as a consequence of the removal of UV and
rapidity divergences in the definition of the TMDs. Despite these
scales are arbitrary scales, they are typically chosen
$\mu=\sqrt{\zeta}=Q$. Therefore, for all practical purposes their
presence is fictitious.

The computation-intensive part of Eq.(\ref{eq:crosssection}) has the
form of the integral:
\begin{equation}\label{eq:integral}
I_{ij}(x_1,x_2,q_T;\mu,\zeta)=\int\frac{d^2\mathbf{b}}{4\pi} e^{i \mathbf{b}\cdot \mathbf{q}_T} \overline{F}_i(x_1,\mathbf{b};\mu,\zeta) \overline{F}_{j}(x_2,\mathbf{b};\mu,\zeta)\,.
\end{equation}
where $\overline{F}_{i(j)}$ are combinations of evolved TMD PDFs. At
this stage, for convenience, $i$ and $j$ do not coincide with $q$ and
$\bar{q}$ but they are linked through a simple linear
transformation. The integral over the bidimensional impact parameter
\textbf{b} has to be taken. However, $\overline{F}_{i(j)}$ only depend
on the absolute value of \textbf{b}, therefore Eq.~(\ref{eq:integral})
can be written as:
\begin{equation}\label{eq:integral2}
I_{ij}(x_1,x_2,q_T;\mu,\zeta)=\frac12\int_0^\infty db\,b J_0(bq_T)  \overline{F}_i(x_1,b;\mu,\zeta) \overline{F}_{j}(x_2,b;\mu,\zeta)\,.
\end{equation}
where $J_0$ is the zero-th order Bessel function of the first kind
whose integral representation is:
\begin{equation}
J_0(x) = \frac1{2\pi}\int_0^{2\pi} d\theta e^{ix\cos(\theta)}\,.
\end{equation}
The evolved quark TMD PDF $\overline{F}_i$ at the final scales $\mu$
and $\zeta$ is obtained by multiplying the same distribution at the
initial scales $\mu_0$ and $\zeta_0$ by a single evolution factor
$R_q$(\footnote{Note that in Eq.~(\ref{eq:crosssection}) the gluon TMD
  PDF $\overline{F}_g$ is not involved. If also the gluon TMD PDF was
  involved, it would evolve by means of a different evolution factor
  $R_g$.}).  that is:
\begin{equation}\label{eq:evolution}
  \overline{F}_i(x,b;\mu,\zeta) = R_q(\mu_0,\zeta_0\rightarrow \mu,\zeta;b)
  \overline{F}_i(x,b;\mu_0,\zeta_0)\,.
\end{equation}

The initial scale TMD PDFs at small values $b$ can be written as:
\begin{equation}\label{eq:LOconv}
\overline{F}_i(x,b;\mu_0,\zeta_0) = \sum_{j=g,q(\bar{q})}x\int_x^1\frac{dy}{y}C_{ij}(y;\mu_0,\zeta_0)f_j\left(\frac{x}{y},\mu_0\right)\,,
\end{equation}
where $f_j$ are the collinear PDFs (including the gluon) and $C_{ij}$
are the so-called matching functions that are perturbatively
computable and are currently known to NNLO, $i.e.$
$\mathcal{O}(\alpha_s^2)$. If we define:
\begin{equation}
\overline{f}_i\left(x,\mu_0\right) = xf_i\left(x,\mu_0\right)\,,
\end{equation}
Eq.~(\ref{eq:LOconv}) can be written as:
\begin{equation}\label{eq:LOconvMod}
\overline{F}_i(x,b;\mu_0,\zeta_0) =
\sum_{j=g,q(\bar{q})}\int_x^1dy\,C_{ij}(y;\mu_0,\zeta_0)  \overline{f}_i\left(\frac{x}{y},\mu_0\right)\,.
\end{equation}
Putting Eqs.~(\ref{eq:evolution}) and~(\ref{eq:LOconvMod}), one finds:
\begin{equation}\label{eq:pertTMD}
  \overline{F}_i(x,b;\mu,\zeta) = R_q(\mu_0,\zeta_0\rightarrow \mu,\zeta;b)
  \sum_{j=g,q(\bar{q})}\int_x^1dy\,C_{ij}(y;\mu_0,\zeta_0)  \overline{f}_i\left(\frac{x}{y},\mu_0\right)\,.
\end{equation}

Matching and evolution are affected by non-perturbative effects that
become relevant at large $b$. In order to account for such effects,
one usually introduces a phenomenological function $f_{\rm NP}$. In
the traditional approach (CSS~\cite{Collins:2011zzd}), the $b$-space
TMDs get a multiplicative correction that does not depend on the
flavour. In addition, the perturbative content of the TMDs is smoothly
damped away at large $b$ by introducing the so-called
$b_*$-prescription:
\begin{equation}\label{eq:LOconvNP1}
  \overline{F}_i(x,b;\mu,\zeta) \rightarrow \overline{F}_i(x,b_*(b);\mu,\zeta) f_{\rm NP}(x,b,\zeta)\,,
\end{equation}
where $b_*\equiv b_*(b)$ is a monotonic function of the impact
parameter $b$ such that:
\begin{equation}
  \lim_{b\rightarrow 0}
  b_*(b) = b_{\rm min}\quad\mbox{and}\quad\lim_{b\rightarrow \infty}
  b_*(b) = b_{\rm max}\,,
\end{equation}
being $b_{\rm min}$ and $b_{\rm max}$ constant values both in the
perturbative region. Including the non-perturbative function,
Eq.~(\ref{eq:integral2}) becomes:
\begin{equation}\label{eq:integral3}
\begin{array}{l}
\displaystyle I_{ij}(x_1,x_2,q_T;\mu,\zeta) = \displaystyle \int_0^\infty db\,J_0(bq_T)\left[\frac{b}2 
\overline{F}_i(x_1,b_*(b);\mu,\zeta) \overline{F}_{j}(x_2,b_*(b);\mu,\zeta) f_{\rm NP}(x_1,b,\zeta)
  f_{\rm NP}(x_2,b,\zeta) \right]\\
\\
\displaystyle =\frac{1}{q_T}\int_0^\infty d\bar{b}\,J_0(\bar{b})\left[\frac{\bar{b}}{2q_T} 
\overline{F}_i(x_1,b_*\left(\frac{\bar{b}}{q_T}\right);\mu,\zeta) \overline{F}_{j}(x_2,,b_*\left(\frac{\bar{b}}{q_T}\right);\mu,\zeta) f_{\rm NP}\left(x_1,\frac{\bar{b}}{q_T},\zeta\right)
  f_{\rm NP}\left(x_2,\frac{\bar{b}}{q_T},\zeta\right) \right]
\,.
\end{array}
\end{equation}
Eq.~(\ref{eq:integral3}) is a Hankel tranform and can be efficiently
computed using the so-called Ogata quadrature~\cite{Ogata:quadrature}.
Effectively, the computation of the integral in
Eq.~(\ref{eq:integral}) is achieved through a weighted sum:
\begin{equation}\label{eq:ogataquadrature}
\begin{array}{rcl}
I_{ij}(x_1,x_2,q_T;\mu,\zeta) &\simeq& \displaystyle
                                                     \frac1{q_T}\sum_{n=1}^N
                                                     \frac{w_n^{(0)}z_n^{(0)}}{2q_T} 
\overline{F}_i\left(x_1,b_*\left
                                       (\frac{z_n^{(0)}}{q_T}\right);\mu,\zeta\right) \overline{F}_j\left(x_2,b_*\left (\frac{z_n^{(0)}}{q_T}\right);\mu,\zeta\right)\\
\\
&\times&\displaystyle f_{\rm NP}\left(x_1,\frac{z_n^{(0)}}{q_T},\zeta\right)
  f_{\rm NP}\left(x_2,\frac{z_n^{(0)}}{q_T},\zeta\right)\,,
\end{array}
\end{equation}
where the unscaled coordinates $z_n^{(0)}$ and the weights $w_n^{(0)}$
can be precomputed in terms of the zero's of the Bessel function $J_0$
and one single parameter (see Ref.~\cite{Ogata:quadrature} for more
details, specifically Eqs.~(5.1) and~(5.2) or
Appendix~\ref{app:OgataQuadrature} for the relevant formula to compute
the unscaled coordinates and the weights)\footnote{The superscript 0
  in $z_n^{(0)}$ and $w_n^{(0)}$ indicates that here we are performing
  a Hankel tranform that involves the Bessel function of degree zero
  $J_0$. This is useful in view of the next section in which the
  integration over $q_T$ will give rise to a similar Hankel transform
  with $J_0$ replaced by $J_1$. Also in that case the Ogata quadrature
  algorithm can be applied but coordinates and weights will be
  different.}. Based on the (empirically verified) assumption that the
absolute value of each term in the sum in the r.h.s. of
Eq.~(\ref{eq:ogataquadrature}) is smaller than that of the preceding
one, the truncation number $N$ is chosen dynamically in such a way
that the $(N+1)$-th term is smaller in absolute value than a
user-defined cutoff relatively to the sum of the preceding $N$ terms.

Eq.~(\ref{eq:ogataquadrature}) factors out the non-perturbative part
of the calculation represented by $f_{\rm NP}$ from the perturbative
content. This is done on purpose to devise a method in which the
perturbative content is precomputed and numerically convoluted with
the non-perturbative functions \textit{a posteriori}. This is
convenient in view of a fit of the function $f_{\rm NP}$.

As customary in QCD, the most convenient basis for the matching in
Eq.~(\ref{eq:LOconv}) is the so-called ``evolution'' basis
(\textit{i.e.} $\Sigma$, $V$, $T_3$, $V_3$, etc.). In fact, in this
basis the operator matrix $C_{ij}$ is almost diagonal with the only
exception of crossing terms that couple the gluon and the singlet
$\Sigma$ distributions. As a consequence, this is the most convenient
basis for the computation of $I_{ij}$. On the other hand, TMDs in
Eq.~(\ref{eq:crosssection}) appear in the so-called ``physical'' basis
(\textit{i.e.} $d$, $\bar{d}$, $u$, $\bar{u}$, etc.). Therefore, we
need to rotate $F_{i(j)}$ from the evolution basis, over which the
indices $i$ and $j$ run, to the physical basis. This is done by means
of an appropriate constant matrix $T$, so that:
\begin{equation}\label{eq:lumiInterRot}
\overline{F}_{q}(x_1,b;\mu,\zeta)= \sum_{i}T_{qi}F_{i}(x_1,b;\mu,\zeta)\,,
\end{equation}
and similarly for $\overline{F}_{\bar{q}}$. Putting all pieces
together, one can conveniently write the cross section in
Eq.~(\ref{eq:crosssection}) as:
\begin{equation}\label{eq:fastdiffxsec}
  \frac{d\sigma}{dQ dy dq_T} \simeq\sum_{n=1}^N w_n^{(0)} \frac{z_n^{(0)}}{q_T}S\left(x_1,x_2,\frac{z_n^{(0)}}{q_T};\mu,\zeta\right) f_{\rm NP}\left(x_1,\frac{z_n^{(0)}}{q_T},\zeta\right) f_{\rm NP}\left(x_2,\frac{z_n^{(0)}}{q_T},\zeta\right)\,,
\end{equation}
with:
\begin{equation}\label{eq:pertfact}
S(x_1,x_2,b;\mu,\zeta) =\frac{8\pi\alpha^2}{9 Q^3}
    H(Q,\mu) \sum_q C_q(Q) \left[
\overline{F}_q\left(x_1,b_*(b);\mu,\zeta \right)\right] \left[
\overline{F}_{\bar{q}}\left(x_2,b_*(b);\mu,\zeta \right)\right]\,.
\end{equation}
Eq.~(\ref{eq:fastdiffxsec}) allows one to precompute the weights $S$
in such a way that the differential cross section in
Eq.~(\ref{eq:crosssection}) can be computed as a simple weighted sum
of the non-perturbative contribution. A misleading aspect of
Eq.~(\ref{eq:pertfact}) is the fact that $S$ has five arguments. In
actual facts, $S$ only depends on three independent variables. The
reason is that $\mu$ and $\zeta$ are usually taken to be proportional
to $Q$ by a constant factor. In addition $x_1$ and $x_2$ depend on $Q$
and $y$ through Eq.~(\ref{eq:Bjorkenx12}). Therefore, the full
dependence on the kinematics of the final state of
Eq.~(\ref{eq:crosssection}) can be specified by $Q$, $y$ and $q_T$.

\section{Integrating over the final-state kinematic variables}

Despite Eq.~(\ref{eq:fastdiffxsec}) provides a powerful tool for a
fast computation of cross sections, it is often not sufficient to
allow for a direct comparison to experimental data. The reason is that
experimental measurements of differential distributions are usually
delivered as integrated over finite regions of the final-state
kinematic phase space. In other words, experiments measure quantities
like:
\begin{equation}\label{eq:Intcrosssection}
\widetilde{\sigma}=\int_{Q_{\rm min}}^{Q_{\rm max}}dQ \int_{y_{\rm min}}^{y_{\rm max}}dy \int_{q_{T,\rm min}}^{q_{T,\rm max}}dq_T\left[\frac{d\sigma}{dQ dy dq_T} \right]\,.
\end{equation}
As a consequence, in order to guarantee performance, we need to
include the integrations above in the precomputed factors. 

\subsection{Integrating over $q_T$}

The integration over bins in $q_T$ can be carried out analytically
exploiting the following property of Bessel's function:
\begin{equation}\label{eq:besselproperty}
\int dx\,x J_0(x) = xJ_1(x)\quad\Rightarrow\quad \int_{x_1}^{x_2}
dx\,x J_0(x) = x_2J_1(x_2) - x_1J_1(x_1)\,.
\end{equation}
To see it, we observe that the differential cross section in
Eq.~(\ref{eq:crosssection}) has the following structure:
\begin{equation}
  \frac{d\sigma}{dQ dy dq_T} \propto \int_0^\infty db\, q_T  J_0(bq_T)\dots
\end{equation}
where the ellipses indicate terms that do not depend on
$q_T$. Therefore, using Eq.~(\ref{eq:besselproperty}) we find:
\begin{equation}
\begin{array}{l}
\displaystyle \int_{q_{T,\rm min}}^{q_{T,\rm
  max}}dq_T\left[\frac{d\sigma}{dQ dy dq_T} \right] \propto \int_0^\infty db\,
  \int_{q_{T,\rm min}}^{q_{T,\rm
  max}} dq_T\,   q_T J_0(bq_T)\dots= \\
\\
\displaystyle \int_0^\infty \frac{db}{b^2}\,
  \int_{bq_{T,\rm min}}^{bq_{T,\rm
  max}} dx\,   x J_0(x)\dots=\int_0^\infty \frac{db}{b}\left[q_{T,\rm
  max}J_1(bq_{T,\rm max}) - q_{T,\rm
  min}J_1(bq_{T,\rm min})\right]\dots\,.
\end{array}
\end{equation}
Therefore, defining:
\begin{equation}
K(q_T) \equiv \int dq_T\left[\frac{d\sigma}{dQ dy dq_T} \right]
\end{equation}
as the indefinite integral over $q_T$ of the cross section in
Eq.~(\ref{eq:crosssection}), we have that:
\begin{equation}\label{eq:primitive}
\int_{q_{T,\rm min}}^{q_{T,\rm
  max}}dq_T\left[\frac{d\sigma}{dQ dy dq_T} \right] = K(Q,y,q_{T,\rm max})
- K(Q,y,q_{T,\rm min})\,,
\end{equation}
with:
\begin{equation}
\begin{array}{l}
 \displaystyle K(Q,y,q_T) =
  \frac{8\pi\alpha^2q_T}{9 Q^3} H(Q,\mu) \\
\\
\displaystyle \times
  \int_0^\infty db\, J_1(bq_T) \sum_q C_q(Q)\overline{F}_q(x_1,b;\mu,\zeta) \overline{F}_{\bar{q}}(x_2,b;\mu,\zeta) f_{\rm NP}(x_1,b,\zeta)
  f_{\rm NP}(x_2,b,\zeta)\,,
\end{array}
\end{equation}
that can be computed using the Ogata quadrature as:
\begin{equation}
  K(Q,y,q_T) \simeq \sum_{n=1}^N w_n^{(1)} S\left(x_1,x_2,\frac{z_n^{(1)}}{q_T};\mu,\zeta\right) f_{\rm NP}\left(x_1,\frac{z_n^{(1)}}{q_T},\zeta\right) f_{\rm NP}\left(x_2,\frac{z_n^{(1)}}{q_T},\zeta\right)\,,
\end{equation}
with $S$ defined in Eq.~(\ref{eq:pertfact}). The unscaled coordinates
$z_n^{(1)}$ and the weights $w_n^{(1)}$ can again be precomputed and
stored in terms of the zero's of the Bessel function
$J_1$. Eq.~(\ref{eq:primitive}) reduces the integration in $q_T$ to a
calculation completely analogous to the unintegrated cross
section. This is particularly convenient because it avoids the
computation a numerical integration.

\subsection{On the position of the peak of the $q_T$ distribution}

It is interesting at this point to take a short detour to discuss the
position of the peak on the distribution in $q_T$ of the cross section
in Eq.~(\ref{eq:crosssection}). The peak can be located by setting the
derivative in $q_T$ of the cross section equal to zero. To do so, we
use another property of Bessel's functions:
\begin{equation}
\frac{dJ_0(x)}{dx} = -J_1(x)\,.
\end{equation}
Using this relation, it is easy to see that:
\begin{equation}
\begin{array}{l}
\displaystyle 0 = \frac{d}{dq_T}  \left[\frac{d\sigma}{dQ dy dq_T}\right]
  =\\
\\
\displaystyle  \frac{8\pi\alpha^2}{9 Q^3} H(Q,\mu) 
  \int_0^\infty db\,b \left[J_0(bq_T) -bq_TJ_1(bq_T)\right] 
\sum_q C_q(Q)\overline{F}_q(x_1,b_*(b);\mu,\zeta)
  \overline{F}_{\bar{q}}(x_2,b_*(b);\mu,\zeta)\\
\\
\displaystyle \times f_{\rm NP}(x_1,b,\zeta)
  f_{\rm NP}(x_2,b,\zeta)\,,
\end{array}
\end{equation}
that is equivalent to require that:
\begin{equation}
  \int_0^\infty db\,b \left[J_0(bq_T) -bq_TJ_1(bq_T)\right] 
  \sum_q C_q(Q)\overline{F}_q(x_1,b_*(b);\mu,\zeta) \overline{F}_{\bar{q}}(x_2,b_*(b);\mu,\zeta) f_{\rm NP}(x_1,b,\zeta)
  f_{\rm NP}(x_2,b,\zeta) = 0\,.
\end{equation}
The integral above can be solved numerically using the technique
discussed above and the value of $q_T$ that satisfies this equation
represents the position of the peak of the $q_T$ distribution.

\subsection{Integrating over $Q$ and $y$}

As a final step, we need to perform the integrals over $Q$ and $y$
defined in Eq.~(\ref{eq:Intcrosssection}). To compute these integrals
we can only rely on numerical methods. Having reduced the integration
in $q_T$ to the difference of the two terms in the r.h.s. of
Eq.~(\ref{eq:primitive}), we can concentrate on integrating the
function $K$ over $Q$ and $y$ for a fixed value of $q_T$:
\begin{equation}
\widetilde{K}(q_T)=\int_{Q_{\rm min}}^{Q_{\rm max}}dQ \int_{y_{\rm
    min}}^{y_{\rm max}}dy\,K(Q,y,q_T)\,,
\end{equation}
such that:
\begin{equation}
  \widetilde{\sigma} = \widetilde{K} (q_{T,\rm max})- \widetilde{K} (q_{T,\rm min})\,.
\end{equation}
To this purpose, it is convenient to make explicit the dependence of
$x_1$ and $x_2$ on $Q$ and $y$ using Eq.~(\ref{eq:Bjorkenx12}). In
addition, for the sake of simplicity we will identify the scales $\mu$
and $\sqrt{\zeta}$ with $Q$ (possible scale variations can be easily
reinstated at a later stage) and thus drop one of the arguments from
the TMD distributions $\overline{F}$ and from the hard factor $H$.
This yields:
\begin{equation}\label{eq:finalintegral}
\begin{array}{rcl}
\displaystyle  \widetilde{K}(q_T) &=& \displaystyle \frac{8\pi q_T}{9} \int_0^\infty db\, J_1(bq_T)
  \int_{Q_{\rm min}}^{Q_{\rm max}}
  \frac{dQ}{Q^3} \alpha^2(Q) H(Q) \\
\\
&\times& \displaystyle 
                         \int_{e^{y_{\rm
    min}}}^{e^{y_{\rm max}}}\frac{d\xi}{\xi} \sum_q C_q(Q)\overline{F}_q\left(\frac{Q}{\sqrt{s}}\xi,b_*(b);Q\right)
                         \overline{F}_{\bar{q}}\left(\frac{Q}{\sqrt{s}}\frac1{\xi},b_*(b);Q\right) \\
\\
&\times& \displaystyle f_{\rm NP}\left(\frac{Q}{\sqrt{s}}\xi,b;Q\right)
  f_{\rm NP}\left(\frac{Q}{\sqrt{s}}\frac1{\xi},b;Q\right)\,,
\end{array}
\end{equation}
where we have performed the change of variable $e^{y} = \xi$. Now we
define one grid in $\xi$, $\{\xi_\alpha\}$ with
$\alpha=0,\dots,N_\xi$, and one grid in $Q$, $\{Q_\tau\}$ with
$\tau=0,\dots,N_Q$, each of which with a set of interpolating
functions $\mathcal{I}$ associated. In addition, the grids are such
that: $\xi_0 = e^{y_{\rm min}}$ and $\xi_{N_\xi} = e^{y_{\rm max}}$,
and $Q_0 = Q_{\rm min}$ and $Q_{N_Q} = Q_{\rm max}$. More details on
the interpolation procedure are presented in
Appendix~\ref{app:LagrangeInterpolation}. This allows us to interpolate
the pair of functions $f_{\rm NP}$ in Eq.~(\ref{eq:finalintegral}) for
generic values of $\xi$ and $Q$ as:
\begin{equation}\label{eq:interpolation}
f_{\rm NP}\left(\frac{Q}{\sqrt{s}}\xi,b;Q\right) f_{\rm NP}\left(\frac{Q}{\sqrt{s}}\frac1{\xi},b;Q\right) \simeq \sum_{\alpha=0}^{N_\xi}\sum_{\tau=0}^{N_Q}\mathcal{I}_\alpha(\xi)\mathcal{I}_\tau(Q) f_{\rm NP}\left(\frac{Q_\tau}{\sqrt{s}}\xi_\alpha,b;Q_\tau\right) f_{\rm NP}\left(\frac{Q_\tau}{\sqrt{s}}\frac1{\xi_\alpha},b;Q_\tau\right)\,.
\end{equation}
Plugging the equation above into Eq.~(\ref{eq:finalintegral}) we
obtain:
\begin{equation}
\begin{array}{rcl}
\displaystyle  \widetilde{K}(q_T) &\simeq& \displaystyle \frac{8\pi q_T}{9} \int_0^\infty db\, J_1(bq_T)
  \sum_{\tau=0}^{N_Q}\sum_{\alpha=0}^{N_\xi}\Bigg[\int_{Q_{\rm min}}^{Q_{\rm max}}dQ\,\mathcal{I}_\tau(Q)\, 
  \frac{1}{Q^3} \alpha^2(Q) H(Q) 
  \\
\\
&\times& \displaystyle 
                         \int_{e^{y_{\rm
    min}}}^{e^{y_{\rm max}}}d\xi\,\mathcal{I}_\alpha(\xi)\,\frac{1}{\xi} \sum_q C_q(Q)\overline{F}_q\left(\frac{Q}{\sqrt{s}}\xi,b_*(b);Q\right)
                         \overline{F}_{\bar{q}}\left(\frac{Q}{\sqrt{s}}\frac1{\xi},b_*(b);Q\right)\Bigg] \\
\\
&\times& \displaystyle f_{\rm NP}\left(\frac{Q_\tau}{\sqrt{s}}\xi_\alpha,b;Q_\tau\right) f_{\rm NP}\left(\frac{Q_\tau}{\sqrt{s}}\frac1{\xi_\alpha},b;Q_\tau\right)\,.
\end{array}
\end{equation}
Finally, the integration over $b$ can be performed using the Ogata
quadrature as discussed above, so that:
\begin{equation}
\begin{array}{rcl}
\displaystyle  \widetilde{K}(q_T) &\simeq& \displaystyle \sum_{n=1}^N
  \sum_{\tau=0}^{N_Q}\sum_{\alpha=0}^{N_\xi}\Bigg[\frac{8\pi}{9} w_n^{(1)}\int_{Q_{\rm min}}^{Q_{\rm max}}dQ\,\mathcal{I}_\tau(Q)\, 
  \frac{1}{Q^3} \alpha^2(Q) H(Q) 
\\
\\
&\times& \displaystyle 
                         \int_{e^{y_{\rm
    min}}}^{e^{y_{\rm max}}}d\xi\,\mathcal{I}_\alpha(\xi)\,\frac{1}{\xi} \sum_q C_q(Q)\overline{F}_q\left(\frac{Q}{\sqrt{s}}\xi,b_*\left(\frac{z_n}{q_T}\right);Q\right)
                         \overline{F}_{\bar{q}}\left(\frac{Q}{\sqrt{s}}\frac1{\xi},b_*\left(\frac{z_n}{q_T}\right);Q\right)\Bigg] \\
\\
&\times& \displaystyle f_{\rm NP}\left(\frac{Q_\tau}{\sqrt{s}}\xi_\alpha,\frac{z_n}{q_T};Q_\tau\right) f_{\rm NP}\left(\frac{Q_\tau}{\sqrt{s}}\frac1{\xi_\alpha},\frac{z_n}{q_T};Q_\tau\right)\,.
\end{array}
\end{equation}
In conclusion, if we define:
\begin{equation}\label{eq:weights}
\begin{array}{rcl}
  \displaystyle  W_{n\tau\alpha}(q_T) & \equiv & \displaystyle w_n^{(1)}\frac{8\pi}{9} \int_{Q_{\rm min}}^{Q_{\rm max}}dQ\,\mathcal{I}_\tau(Q)\, 
                                            \frac{\alpha^2(Q)}{Q^3} H(Q) 
                                            \\
  \\
                                 &\times& \displaystyle 
                                          \int_{e^{y_{\rm
                                          min}}}^{e^{y_{\rm max}}}d\xi\,\mathcal{I}_\alpha(\xi)\,\frac{1}{\xi} \sum_q C_q(Q)\overline{F}_q\left(\frac{Q}{\sqrt{s}}\xi,b_*\left(\frac{z_n}{q_T}\right);Q\right)
                                          \overline{F}_{\bar{q}}\left(\frac{Q}{\sqrt{s}}\frac1{\xi},b_*\left(\frac{z_n}{q_T}\right);Q\right)\,,
\end{array}
\end{equation}
the quantity $\widetilde{K}(q_T)$ can be computed as:
\begin{equation}\label{eq:finalinterpolated}
\widetilde{K}(q_T) \simeq \sum_{n=1}^N
  \sum_{\tau=0}^{N_Q}\sum_{\alpha=0}^{N_\xi} W_{n\tau\alpha}(q_T) f_{\rm NP}\left(\frac{Q_\tau}{\sqrt{s}}\xi_\alpha,\frac{z_n}{q_T};Q_\tau\right) f_{\rm NP}\left(\frac{Q_\tau}{\sqrt{s}}\frac1{\xi_\alpha},\frac{z_n}{q_T};Q_\tau\right)\,.
\end{equation}
The advantage of Eq.~(\ref{eq:finalinterpolated}) is that the weights
$W_{n\alpha\tau}$, that clearly depend on $q_T$ but also on the
intervals $[Q_{\rm min}:Q_{\rm max}]$ and $[y_{\rm min}:y_{\rm max}]$,
can be precomputed once and for all for each of the experimental
points included in a fit and used to determine the function
$f_{\rm NP}$.  This provides a fast tool for the computation of
predictions that makes the extraction of the non-perturbative part of
the TMDs much easier.

\subsection{Narrow-width approximation}

A possible alternative to the numerical integration in $Q$ when the
integration region includes the $Z$-peak region is the so-called
narrow-width approximation (NWA). In the NWA one assumes that the
width of the $Z$ boson, $\Gamma_Z$, is much smaller than its mass,
$M_Z$. This way one can approximate the peaked behaviour of the
couplings $C_q(Q)$ around $Q=M_Z$ with a $\delta$-function,
\textit{i.e.}  $C_q(Q)\sim \delta(Q^2-M_Z^2)$. Therefore, the
integration over $Q$ can be done analytically. The exact structure of
the electroweak couplings is the following:
\begin{equation}\label{eq:fullcoup}
C_q(Q) = e_q^2 - 2 e_q V_q V_e \chi_1(Q) + (V_e^2 + A_e^2)(V_q^2 + A_q^2)\chi_2(Q)\,,
\end{equation}
with:
\begin{equation}
\begin{array}{l}
\displaystyle \chi_1(Q) = \frac{1}{4 \sin^2\theta_W \cos^2\theta_W } \frac{Q^2 ( Q^2 -  M_Z^2 )}{ (Q^2 - M_Z^2)^2 + M_Z^2 \Gamma_Z^2} \,,\\
\displaystyle \chi_2(Q) = \frac{1}{16 \sin^4\theta_W\cos^4\theta_W} \frac{Q^4}{ (Q^2 - M_Z^2)^2 + M_Z^2 \Gamma_Z^2} \,.
\end{array}
\end{equation}
In the limit $\Gamma_Z/M_Z\rightarrow 0$, the leading contribution to
the coupling in Eq.~(\ref{eq:fullcoup}) comes from the region
$Q\simeq M_Z$ and is that proportional to $\chi_2$:
\begin{equation}\label{eq:partlead}
C_q(Q) \simeq (V_e^2 + A_e^2)(V_q^2 + A_q^2)\chi_2(Q)\,,\quad Q\simeq M_Z\,.
\end{equation}
In addition, in this limit one can show that:
\begin{equation}\label{eq:breitwigner}
\frac{1}{ (Q^2 - M_Z^2)^2 + M_Z^2 \Gamma_Z^2}\rightarrow
\frac{\pi}{M_Z\Gamma_Z}\delta(Q^2-M_Z^2) = \frac{\pi}{2M_Z^2\Gamma_Z}\delta(Q-M_Z)\,.
\end{equation}
Therefore, considering that:
\begin{equation}
\Gamma_Z = \frac{\alpha M_Z}{\sin^2\theta_W \cos^2\theta_W}\,,
\end{equation}
the electroweak couplings in the NWA have the following form:
\begin{equation}\label{eq:partlead}
  C_q(Q) \simeq \frac{\pi M_Z (V_e^2 + A_e^2)(V_q^2 + A_q^2) }{32 \alpha
    \sin^2\theta_W\cos^2\theta_W} \delta(Q-M_Z)=\widetilde{C}_q(Q) \delta(Q-M_Z)\,.
\end{equation}
Therefore, using Eq.~(\ref{eq:partlead}) the integral of the cross
section over $Q$ under the condition that
$Q_{\rm min}<M_Z<Q_{\rm max}$ has the consequence of adjusting the
couplings and of setting $Q=M_Z$ in the computation. This yields:
\begin{equation}
\int_{Q_{\rm min}}^{Q_{\rm max}}dQ\,\frac{d\sigma}{dQ dy dq_T} =
 \frac{16\pi\alpha^2q_T}{9 M_Z^3} H(M_Z,M_Z) \sum_q \widetilde{C}_q(M_Z)
  I_{q\bar{q}}(x_1,x_2,q_T;M_Z,M_Z^2)\,,
\end{equation}
where we are also assuming that $\mu=\sqrt{\zeta}=M_Z$. As a final
step, one may want to let the $Z$ boson decay into leptons. At leading
order in the EW sector and assuming an equal decay rate for electrons,
muons, and tauons, this can be done by multiplying the cross section
above by three times the branching ratio for the $Z$ decaying into any
pair of leptons, $3\mbox{Br}(Z\rightarrow \ell^+\ell^-)$.

\appendix

\section{Ogata quadrature}\label{app:OgataQuadrature}

In this section we limit ourselves to write the formulas for the
computation of the unscaled coordinates $z_n^{(\nu)}$ and weights
$w_n^{(\nu)}$ required to compute the following integral:
\begin{equation}\label{eq:OgataQuadMast}
I_\nu(q_T)=\int_0^\infty db J_\nu(bq_T) f\left(b\right) =
\frac1{q_T}\int_0^\infty d\bar{b} J_\nu(\bar{b})
f\left(\frac{\bar{b}}{q_T}\right) \simeq
\frac{1}{q_T}\sum_{n=1}^\infty
w_n^{(\nu)}f\left(\frac{z_n^{(\nu)}}{q_T}\right)\quad \nu =0,1,\dots\,,
\end{equation}
using the Ogata-quadrature algorithm. More details can be found in
Ref.~\cite{Ogata:quadrature}. There relevant formulas are:
\begin{equation}
\begin{array}{l}
\displaystyle z_n^{(\nu)} = \frac{\pi}{h}  \psi\left(\frac{h\xi_{\nu
  n}}{\pi}\right)\,,\\
\\
\displaystyle w_n^{(\nu)}  = \pi\frac{Y_\nu(\xi_{\nu
  n})}{J_{\nu+1}(\xi_{\nu n})}  J_\nu(z_n^{(\nu)})  \psi'\left(\frac{h\xi_{\nu
  n}}{\pi}\right)\,.
\end{array}
\end{equation}
where:
\begin{itemize}
\item $h$ is a free parameter of the algorithm that hat to be
  typically small (we choose $h = 10^{-3}$),
\item $\xi_{\nu n}$ are the zero's of $J_\nu$, \textit{i.e.}
  $J_\nu(\xi_{\nu n}) = 0$ $\forall\, n$,
\item $J_\nu$ and $Y_\nu$ are the Bessel functions of first and second
  kind, respectively, of degree $\nu$,
\item $\psi$ is the following function:
\begin{equation}
\psi(t) = t\tanh\left(\frac{\pi}{2}\sinh t\right)
\end{equation}
and its derivative:
\begin{equation}
\psi'(t) =  \frac{\pi t \cosh t + \sinh( \pi \sinh t ) }{1 +
  \cosh( \pi \sinh t  ) }\,.
\end{equation}
\end{itemize}

\section{Lagrange interpolation}\label{app:LagrangeInterpolation}

Just for the record, it is useful to derive a general expression for
the Lagrange interpolating functions $\mathcal{I}$ introduced in
Eq.~(\ref{eq:interpolation}) and used to interpolate the
non-perturbative functions $f_{\rm NP}$. More, importantly, we need to
understand how these functions behave upon integration.

Suppose one wants to interpolate the test function $g$ in the point
$x$ using a set of Lagrange polynomials of degree $k$ of. This
requires a subset of $k+1$ consecutive points on an interpolation
grid, say $\{x_{\alpha},\dots,x_{\alpha+k}\}$. The relative position
between the point $x$ and the subset of points used for the
interpolation is arbitrary. It is convenient to choose the subset of
points such that $x_\alpha < x \leq x_{\alpha+k}$.\footnote{In fact,
  it is not even necessary to impose the constraint
  $x_\alpha < x \leq x_{\alpha+k}$.  In case this relation is not
  fulfilled one usually refers to \textit{extrapolation} rather than
  \textit{interpolation}. If not necessary, this option is typically
  not convenient because it may lead to a substantial deterioration in
  the accuracy with which $g(x)$ is determined.}  However, the
ambiguity remains because there are $k$ possible choices according to
whether $x_\alpha < x \leq x_{\alpha+1}$, or
$x_{\alpha+1} < x \leq x_{\alpha+2}$, and so on.

In order to determine the exact form of the interpolation functions
$\mathcal{I}$, let us see how to derive
eq.~(\ref{eq:interpolation}). Using the standard Lagrange
interpolation procedure, we can approximate the function $g$ in $x$
as:
\begin{equation}\label{particularCase}
g(x) = \sum_{i=0}^k\ell_i^{(k)}(x)g(x_{\alpha+i})\,,
\end{equation}
where $\ell_i^{(k)}$ is the $i$-th Lagrange polynomial of degree $k$
which can be written as:
\begin{equation} \ell_i^{(k)}(x) = \prod^{k}_{m=0,m\ne
i}\frac{x-x_{\alpha+m}}{x_{\alpha+i}-x_{\alpha+m}}\,.
\end{equation}
We now assume that:
\begin{equation}\label{eq:assumption1}
x_{\alpha} < x \leq x_{\alpha+1}\,,
\end{equation}
Eq.~(\ref{particularCase}) becomes:
\begin{equation}\label{particularCaseTheta}
  g(x) =
  \theta(x-x_{\alpha})\theta(x_{\alpha+1}-x)\sum_{i=0}^k
  g(x_{\alpha+i})\prod^{k}_{m=0,m\ne
    i}\frac{x-x_{\alpha+m}}{x_{\alpha+i}-x_{\alpha+m}}\,.
\end{equation}

In order to make Eq.~(\ref{particularCaseTheta}) valid for all values
of $\alpha$, one just has to sum over all $N_x$ intervals of the
\textit{global} interpolation grid $\{x_0,\dots,x_{N_x}\}$, that is:
\begin{equation}\label{generalCase}
  g(x) =
  \sum_{\alpha=0}^{N_x-1}\theta(x-x_{\alpha})\theta(x_{\alpha+1}-x)\sum_{i=0}^k
  g(x_{\alpha+i})\prod^{k}_{m=0,m\ne
    i}\frac{x-x_{\alpha+m}}{x_{\alpha+i}-x_{\alpha+m}}\,,
\end{equation}

Defining $\beta=\alpha+i$, we can rearrange the equation above as:
\begin{equation}\label{generalCase2}
  g(x) =
  \sum_{\beta=0}^{N_x+k-1}\mathcal{I}_\beta^{(k)}(x) g(x_{\beta})\,,
\end{equation}
that leads us to the definition of the interpolating functions:
\begin{equation}\label{eq:intfunc}
  \mathcal{I}_\beta^{(k)}(x) = \sum_{i=0,i\leq\beta}^k
  \theta(x-x_{\beta-i})\theta(x_{\beta-i+1}-x) \prod^{k}_{m=0,m\ne
    i}\frac{x-x_{\beta-i+m}}{x_{\beta}-x_{\beta-i+m}}\,,
\end{equation}
where the condition $i\leq\beta$ comes from the condition
$\alpha\geq 0$. It is important to observe that the sum in
Eq.~(\ref{generalCase2}) extends up to the $(N_x+k-1)$-th
node. Therefore, the original grid needs to be extended by $k-1$
nodes. However, the range of validity of the interpolation remains
that defined by the original grid, \textit{i.e.}
$x_0 \leq x \leq x_{N_x}$. Finally, it is crucial to realise that the
interpolation function $\mathcal{I}_\beta^{(k)}(x)$ is different from
zero only over a limited interval, specifically:
\begin{equation}\label{eq:limits}
\mathcal{I}_\beta^{(k)}(x) \neq 0\quad \Leftrightarrow\quad
x_{\beta-k}<x < x_{\beta+1}\,.
\end{equation}

In the rest of this document we will stick to the assumption in
Eq.~(\ref{eq:assumption1}). However, before going further, it is
interesting to generalise Eq.~(\ref{eq:assumption1}) to:
\begin{equation}\label{IntAssumptionGen}
  x_{\alpha+t} < x \leq
  x_{\alpha+t+1}\quad\mbox{with}\quad t = 0,\dots,k-1\,,
\end{equation}
such that the interpolation formula becomes:
\begin{equation}\label{MoreGeneralCase}
  g(x) =
  \sum_{\alpha=-t}^{N_x-t-1}\theta(x-x_{\alpha+t})\theta(x_{\alpha+t+1}-x)\sum_{i=0}^k
  g(x_{\alpha+i})\prod^{k}_{m=0,m\ne
    i}\frac{x-x_{\alpha+m}}{x_{\alpha+i}-x_{\alpha+m}}\,,
\end{equation}
that can be rearranged as:
\begin{equation}\label{generalCase3} g(x) =
\sum_{\beta=-t}^{N_x+k-t-1}\mathcal{I}_{\beta,t}^{(k)}(x) g(x_{\beta})\,,
\end{equation}
with:
\begin{equation}\label{eq:generalisedintfuncs}
\mathcal{I}_{\beta,t}^{(k)}(x) = \sum_{i=0,i\leq\beta}^k
\theta(x-x_{\beta-i+t})\theta(x_{\beta-i+t+1}-x) \prod^{k}_{m=0,m\ne
i}\frac{x-x_{\beta-i+m}}{x_{\beta}-x_{\beta-i+m}}\,,
\end{equation}
being the ``generalised'' interpolation functions.  The generalised
interpolation functions can be used to overcome the ``drawback'' of
requiring $k-1$ additional nodes on the interpolation grid. In
practice, given the grid $\{x_0,\dots,x_{N_x}\}$, one can tune $t$
according to the position of $x$ on the grid. More specifically, one
can choose $t$ in such a way that $\beta+t$ in
Eq.~(\ref{eq:generalisedintfuncs}) never exceeds $N_x$.

Now suppose we want to compute the following intergral:
\begin{equation}
I_1 = \int_{x_0}^{x_{N_x}}dx\,g(x)f(x)\,,
\end{equation}
where $f$ is some other function that we don't want to
interpolate. Using Eqs.~(\ref{generalCase2}) and~(\ref{eq:limits}) we
finally have that:
\begin{equation}
  I_1 = \sum_{\beta=0}^{N_x+k-1} W_\beta g(x_{\beta})\,,
\end{equation}
with:
\begin{equation}\label{eq:monodim}
W_\beta = \int_{x_{{\rm max}(0,\beta-k)}}^{x_{{\rm min}(N_x,\beta+1)}}dx \,\mathcal{I}_\beta^{(k)}(x)f(x)\,.
\end{equation}
The equation above can be easily generalised to a bidimensional
integral as:
\begin{equation}
I_2 = \int_{x_0}^{x_{N_x}}dx \int_{y_0}^{y_{N_y}}dy\,g(x,y)f(x,y) = \sum_{\alpha=0}^{N_x+k-1} \sum_{\beta=0}^{N_y+l-1} W_{\alpha\beta} g(x_{\alpha},y_{\beta})\,,
\end{equation}
with:
\begin{equation}\label{eq:bidim}
W_{\alpha\beta} = \int_{x_{{\rm max}(0,\alpha-k)}}^{x_{{\rm
      min}(N_x,\alpha+1)}}dx \int_{y_{{\rm max}(0,\beta-k)}}^{y_{{\rm
      min}(N_y,\beta+1)}}dy \,\mathcal{I}_\alpha^{(k)}(x)\,\mathcal{I}_\beta^{(l)}(y)\,f(x,y)\,.
\end{equation}
This formalism nicely applies to the integral in $Q$ and $\xi=e^{y}$
discussed above in Eq.~(\ref{eq:weights}). In view of a numerical
implementation, it is worth noticing that the functions $\mathcal{I}$
are piecewise. In particular, while these functions are continuos in
correspondence of the nodes of the grid, their first derivative is
not. As a consequence, the result of the numerical integrals in
Eqs.~(\ref{eq:monodim}) and~(\ref{eq:bidim}) may be inaccurate. To
overcome this problem, it is sufficient to split the integrals in
sub-integrals over the intervals delimited by two consecutive
nodes. Using Eq.~(\ref{eq:limits}), it is easy to see that, for an
interpolation of degree $k$, one needs to do $k+1$ integrals over the
intervals included between the $(\beta-k)$-th and the $(\beta+1)$-th
node.







\begin{thebibliography}{alp}

%\cite{Scimemi:2017etj}
\bibitem{Scimemi:2017etj}
  I.~Scimemi and A.~Vladimirov,
  %``Analysis of vector boson production within TMD factorization,''
  arXiv:1706.01473 [hep-ph].
  %%CITATION = ARXIV:1706.01473;%%
  %2 citations counted in INSPIRE as of 24 Oct 2017

%\cite{Collins:2011zzd}
\bibitem{Collins:2011zzd}
  J.~Collins,
  %``Foundations of perturbative QCD,''
  Camb.\ Monogr.\ Part.\ Phys.\ Nucl.\ Phys.\ Cosmol.\  {\bf 32} (2011) 1.
  %%CITATION = CMPCE,32,1;%%
  %327 citations counted in INSPIRE as of 21 Oct 2018

\bibitem{Ogata:quadrature}
  H.~Ogata,
  ``A Numerical Integration Formula Based on the Bessel Functions,''
  \texttt{http://www.kurims.kyoto-u.ac.jp/$\sim$okamoto/paper/Publ\_RIMS\_DE/41-4-40.pdf}

\end{thebibliography}

\end{document}
