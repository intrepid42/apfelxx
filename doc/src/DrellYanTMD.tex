\documentclass[10pt,a4paper]{article}
\usepackage{amsmath,amssymb,bm,makeidx,subfigure}
\usepackage[italian,english]{babel}
\usepackage[center,small]{caption}[2007/01/07]
\usepackage{fancyhdr}
\usepackage{color}

\definecolor{blu}{rgb}{0,0,1}
\definecolor{verde}{rgb}{0,1,0}
\definecolor{rosso}{rgb}{1,0,0}
\definecolor{viola}{rgb}{1,0,1}
\definecolor{arancio}{rgb}{1,0.5,0}
\definecolor{celeste}{rgb}{0,1,1}
\definecolor{rosa}{rgb}{1,0.3,0.5}

\oddsidemargin = 12pt
\topmargin = 0pt
\textwidth = 440pt
\textheight = 650pt

\makeindex

\begin{document}

\section{Structure of the observables}

Let us start from Eq.~(2.6) of Ref.~\cite{Scimemi:2017etj}, that is
the fully differential cross section for lepton-pair production in the
region in which the TMD factorisation applies, $i.e.$ $q_T \ll
Q$. After some minor manipulations, it reads:
\begin{equation}\label{eq:crosssection}
  \frac{d\sigma}{dQ dy dq_T} =
  \frac{16\pi\alpha^2q_T}{3N_c Q^3} H(Q,\mu) \sum_q C_q(Q)
  \int\frac{d^2\mathbf{b}}{4\pi} e^{i \mathbf{b}\cdot \mathbf{q}_T} \overline{F}_q(x_1,\mathbf{b};\mu,\zeta) \overline{F}_{\bar{q}}(x_2,\mathbf{b};\mu,\zeta)\,,
\end{equation}
where $Q$, $y$, and $q_T$ are the invariant mass, the rapidity, and
the transverse momentum of the lepton pair, respectively, while
$N_c=3$ is the number of colours, $\alpha$ is the electromagnetic
coupling, $H$ is the appropriate QCD hard factor that can be
perturbatively computed, and $C_q$ are the effective electroweak
charges. In addition, the variables $x_1$ and $x_2$ are functions of
$Q$ and $y$ and are given by:
\begin{equation}\label{eq:Bjorkenx12}
  x_{1,2} = \frac{Q}{\sqrt{s}}e^{\pm y}\,,
\end{equation}
being $\sqrt{s}$ the centre-of-mass energy of the collision. In
Eq.~(\ref{eq:crosssection}) we are using the short-hand notation:
\begin{equation}
\overline{F}_q(x,\mathbf{b};\mu,\zeta) \equiv xF_q(x,\mathbf{b};\mu,\zeta)\,,
\end{equation}
that is convenient for the implementation. The scales $\mu$ and
$\zeta$ are introduced through TMD factorisation to factorise
collinear and rapidity divergences. As usual, despite they are
arbitrary scales, they are typically chosen
$\mu=\sqrt{\zeta}=Q$. Therefore, for all practical purposes their
presence is fictitious.

The computation-intensive part of Eq.(\ref{eq:crosssection}) has the
form of the integral:
\begin{equation}\label{eq:integral}
I_{ij}(x_1,x_2,q_T;\mu,\zeta)=\int\frac{d^2\mathbf{b}}{4\pi} e^{i \mathbf{b}\cdot \mathbf{q}_T} \overline{F}_i(x_1,\mathbf{b};\mu,\zeta) \overline{F}_{j}(x_2,\mathbf{b};\mu,\zeta)\,.
\end{equation}
where $\overline{F}_{i(j)}$ are combinations of evolved TMD PDFs. At
this stage, for convenience, $i$ and $j$ do not coincide with $q$ and
$\bar{q}$ but they are linked through a simple linear
transformation. The integral over the bidimensional impact parameter
\textbf{b} has to be taken. However, $\overline{F}_{i(j)}$ only depend
on the absolute value of \textbf{b}, therefore Eq.~(\ref{eq:integral})
can be written as:
\begin{equation}\label{eq:integral2}
I_{ij}(x_1,x_2,q_T;\mu,\zeta)=\frac12\int_0^\infty db\,b J_0(bq_T)  \overline{F}_i(x_1,b;\mu,\zeta) \overline{F}_{j}(x_2,b;\mu,\zeta)\,.
\end{equation}
where $J_0$ is the zero-th order Bessel function of the first kind
whose integral representation is:
\begin{equation}
J_0(x) = \frac1{2\pi}\int_0^{2\pi} d\theta e^{ix\cos(\theta)}\,.
\end{equation}
The single evolved TMD PDF $\overline{F}_i$ at the final scales $\mu$
and $\zeta$ is obtained by multiplying the same TMD PDF at the initial
scales $\mu_0$ and $\zeta_0$ by a single evolution factor
$R_q$(\footnote{Note that in Eq.~(\ref{eq:crosssection}) the gluon TMD
  PDF $\overline{F}_g$ is not involved. If also the gluon TMD PDF was
  involved, it would evolve by means of a different evolution factor
  $R_g$.}).  that is:
\begin{equation}\label{eq:evolution}
  \overline{F}_i(x,b;\mu,\zeta) = R_q(\mu_0,\zeta_0\rightarrow \mu,\zeta;b)
  \overline{F}_i(x,b;\mu_0,\zeta_0)\,.
\end{equation}

The initial scale TMD PDFs at small values $b$ can be written as:
\begin{equation}\label{eq:LOconv}
\overline{F}_i(x,b;\mu_0,\zeta_0) = \sum_{j=g,q(\bar{q})}x\int_x^1\frac{dy}{y}C_{ij}(y;\mu_0,\zeta_0)f_j\left(\frac{x}{y},\mu_0\right)\,,
\end{equation}
where $f_j$ are the collinear PDFs (including the gluon) and $C_{ij}$
are the so-called matching functions that are perturbatively
computable and are currently known to NNLO, $i.e.$
$\mathcal{O}(\alpha_s^2)$. If we define:
\begin{equation}
\overline{f}_i\left(x,\mu_0\right) = xf_i\left(x,\mu_0\right)\,,
\end{equation}
Eq.~(\ref{eq:LOconv}) can be written as:
\begin{equation}\label{eq:LOconvMod}
\overline{F}_i(x,b;\mu_0,\zeta_0) =
\sum_{j=g,q(\bar{q})}\int_x^1dy\,C_{ij}(y;\mu_0,\zeta_0)  \overline{f}_i\left(\frac{x}{y},\mu_0\right)\,.
\end{equation}
Putting Eqs.~(\ref{eq:evolution}) and~(\ref{eq:LOconvMod}), one finds:
\begin{equation}\label{eq:pertTMD}
  \overline{F}_i(x,b;\mu,\zeta) = R_q(\mu_0,\zeta_0\rightarrow \mu,\zeta;b)
  \sum_{j=g,q(\bar{q})}\int_x^1dy\,C_{ij}(y;\mu_0,\zeta_0)  \overline{f}_i\left(\frac{x}{y},\mu_0\right)\,.
\end{equation}

Matching and the evolution are affected by non-perturbative effects
that become relevant at large $b$. In order to account for such
effects, one usually introduces a phenomenological function
$f_{\rm NP}$. In the traditional approach (CSS~\cite{Collins:2011zzd})
the $b$-space TMDs get a multiplicative correction that does not
depend on the flavour. In addition, the perturbative content of the
TMDs is smoothly damped away at large $b$ by introducing the so-called
$b_*$-prescription:
\begin{equation}\label{eq:LOconvNP1}
  \overline{F}_i(x,b;\mu,\zeta) \rightarrow \overline{F}_i(x,b_*;\mu,\zeta) f_{\rm NP}(x,b,\zeta)\,,
\end{equation}
where $b_*\equiv b_*(b)$ is a monotonic function of the impact
parameter $b$ such that:
\begin{equation}
  \lim_{b\rightarrow 0}
  b_*(b) = b_{\rm min}\quad\mbox{and}\quad\lim_{b\rightarrow \infty}
  b_*(b) = b_{\rm max}\,,
\end{equation}
being $b_{\rm min}$ and $b_{\rm max}$ constant values both in the
perturbative region. Including the non-perturbative function,
Eq.~(\ref{eq:integral2}) becomes:
\begin{equation}\label{eq:integral3}
\begin{array}{l}
\displaystyle I_{ij}(x_1,x_2,q_T;\mu,\zeta) = \displaystyle \int_0^\infty db\,J_0(bq_T)\left[\frac{b}2 
\mathcal{L}_{ij}(x_1,x_2,b_*(b);\mu,\zeta) f_{\rm NP}(x_1,b,\zeta)
  f_{\rm NP}(x_2,b,\zeta) \right]\\
\\
\displaystyle =\frac{1}{q_T}\int_0^\infty d\bar{b}\,J_0(\bar{b})\left[\frac{\bar{b}}{2q_T} 
\mathcal{L}_{ij}\left(x_1,x_2,b_*\left (\frac{\bar{b}}{q_T}\right);\mu,\zeta\right) f_{\rm NP}\left(x_1,\frac{\bar{b}}{q_T},\zeta\right)
  f_{\rm NP}\left(x_2,\frac{\bar{b}}{q_T},\zeta\right) \right]
\,,
\end{array}
\end{equation}
with:
\begin{equation}
\mathcal{L}_{ij} (x_1,x_2,b_*;\mu,\zeta) \equiv
\overline{F}_i(x_1,b_*;\mu,\zeta) \overline{F}_{j}(x_2,b_*;\mu,\zeta) \,.
\end{equation}
Eq.~(\ref{eq:integral3}) is a Hankel tranform and can be efficiently
computed using the so-called Ogata quadrature~\cite{Ogata:quadrature}.
Effectively, the computation of the integral in
Eq.~(\ref{eq:integral}) is achieved through a weighted sum:
\begin{equation}\label{eq:ogataquadrature}
I_{ij}(x_1,x_2,q_T;\mu,\zeta) \simeq
                                                     \frac1{q_T}\sum_{n=1}^N
                                                     \frac{w_n^{(0)}z_n^{(0)}}{2q_T} 
\mathcal{L}_{ij}\left(x_1,x_2,b_*\left (\frac{z_n^{(0)}}{q_T}\right);\mu,\zeta\right) f_{\rm NP}\left(x_1,\frac{z_n^{(0)}}{q_T},\zeta\right)
  f_{\rm NP}\left(x_2,\frac{z_n^{(0)}}{q_T},\zeta\right)\,,
\end{equation}
where the unscaled coordinates $z_n^{(0)}$ and the weights $w_n^{(0)}$
can be precomputed in terms of Bessel functions and one single
parameter (see Ref.~\cite{Ogata:quadrature} for more details,
specifically Eqs.~(5.1) and~(5.2))\footnote{The superscript 0 in
  $z_n^{(0)}$ and $w_n^{(0)}$ indicates that here we are performing a
  Hankel tranform that involves the Bessel function of degree zero
  $J_0$. This is useful in view of the next section in which the
  integration over $q_T$ will give rise to a similar Hankel transform
  with $J_0$ replaced by $J_1$. Also in that case the Ogata quadrature
  algorithm can be applied.}. Based on the (empirically verified)
assumption that the absolute value of each term in the sum in the
r.h.s. of Eq.~(\ref{eq:ogataquadrature}) is smaller than that of the
preceding one, the truncation number $N$ is chosen dynamically in such
a way that the $(N+1)$-th term is smaller in absolute value than a
user-defined cutoff relatively to the sum of the preceding $N$
terms. Eq.~(\ref{eq:ogataquadrature}) factors out the non-perturbative
part of the calculation represented by $f_{\rm NP}$ from the
perturbative content.

As customary in QCD, the most convenient basis for the matching in
Eq.~(\ref{eq:LOconv}) is the so-called ``evolution'' basis
(\textit{i.e.} $\Sigma$, $V$, $T_3$, $V_3$, etc.). In fact, in this
basis the operators matrix $C_{ij}$ is almost diagonal with the only
exception of crossing terms that couple the gluon and the singlet
$\Sigma$ distributions. As a consequence, this is the most convenient
basis for the computation of $I_{ij}$. On the other hand, TMDs in
Eq.~(\ref{eq:crosssection}) are in the so-called ``physical'' basis
(\textit{i.e.} $d$, $\bar{d}$, $u$, $\bar{u}$, etc.). Therefore, we
need to rotate $I_{ij}$ from the evolution basis, over which the
indices $i$ and $j$ run, to the physical basis. This is done by means
of an appropriate constant matrix $T$, so that:
\begin{equation}\label{eq:lumiInterRot}
I_{q\bar{q}}(x_1,x_2,q_T;\mu,\zeta)= \sum_{ij}T_{qi}T_{\bar{q}j}I_{ij}(x_1,x_2,q_T;\mu,\zeta)\,.
\end{equation}
Putting all pieces together, one can conveniently write the cross
section in Eq.~(\ref{eq:crosssection}) as:
\begin{equation}\label{eq:fastdiffxsec}
  \frac{d\sigma}{dQ dy dq_T} =\sum_{n=1}^N w_n^{(0)} S\left(x_1,x_2,\frac{z_n^{(0)}}{q_T};\mu,\zeta\right) f_{\rm NP}\left(x_1,\frac{z_n^{(0)}}{q_T},\zeta\right) f_{\rm NP}\left(x_2,\frac{z_n^{(0)}}{q_T},\zeta\right)\,,
\end{equation}
with:
\begin{equation}\label{eq:pertfact}
\begin{array}{l}
\displaystyle  S(x_1,x_2,b;\mu,\zeta) =
  \frac{16\pi\alpha^2}{3N_c Q^3} H(Q,\mu) \frac{b}{2}\sum_q C_q(Q)
  \sum_{ij}T_{qi}T_{\bar{q}j} 
\mathcal{L}_{ij}\left(x_1,x_2,b_*(b);\mu,\zeta\right)\\
\\
\displaystyle =\frac{16\pi\alpha^2}{3N_c Q^3}
    H(Q,\mu) \frac{b}{2}\sum_q C_q(Q) \left[
\overline{F}_q\left(x_1,b_*(b);\mu,\zeta \right)\right] \left[
\overline{F}_{\bar{q}}\left(x_2,b_*(b);\mu,\zeta \right)\right]\,,
\end{array}
\end{equation}
with:
\begin{equation}
\overline{F}_q(x_1,b;\mu,\zeta) = \sum_{i} T_{qi}\overline{F}_i(x_1,b;\mu,\zeta)\,.
\end{equation}
Eq.~(\ref{eq:fastdiffxsec}) allows one to precompute the weights $S$
in such a way that the differential cross section in
Eq.~(\ref{eq:crosssection}) can be computed as a simple weighted sum
of the non-perturbative contribution. A misleading aspect of
Eq.~(\ref{eq:pertfact}) is the fact that $S$ has five arguments. In
actual facts, $S$ only depends on three independent variables. The
reason is that $\mu$ and $\zeta$ are usually taken to be proportional
to $Q$ by a constant factor. In addition $x_1$ and $x_2$ depend on $Q$
and $y$ through Eq.~(\ref{eq:Bjorkenx12}). Therefore, as expected, the
full dependence on the kinematics of the final state of
Eq.~(\ref{eq:crosssection}) can be specified by $Q$, $y$ and $q_T$ or
alternatively by $x_1$, $x_2$ and $q_T$.

\section{Integrating over the final-state kinematic variables}

Despite Eq.~(\ref{eq:fastdiffxsec}) provides a powerful tool for a
fast computation of cross sections, it is often not sufficient to
allow for a direct comparison to experimental data. The reason is that
experimental measurements of differential distributions are usually
delivered as integrated over finite regions of the final-state
kinematic phase space. In other words, experiments measure quantities
like:
\begin{equation}\label{eq:Intcrosssection}
\widetilde{\sigma}=\int_{Q_{\rm min}}^{Q_{\rm max}}dQ \int_{y_{\rm min}}^{y_{\rm max}}dy \int_{q_{T,\rm min}}^{q_{T,\rm max}}dq_T\left[\frac{d\sigma}{dQ dy dq_T} \right]\,.
\end{equation}
As a consequence, in order to guarantee performance, we need to
include the integrations above in the precomputed factors. 

\subsection{Integrating over $q_T$}

The integration over bins in $q_T$ can be carried out analytically
exploiting the following property of Bessel's function:
\begin{equation}\label{eq:besselproperty}
\int dx\,x J_0(x) = xJ_1(x)\quad\Rightarrow\quad \int_{x_1}^{x_2}
dx\,x J_0(x) = x_2J_1(x_2) - x_1J_1(x_1)\,.
\end{equation}
To see it, we observe that the differential cross section in
Eq.~(\ref{eq:crosssection}) has the following general structure:
\begin{equation}
  \frac{d\sigma}{dQ dy dq_T} \propto \int_0^\infty db\, q_T  J_0(bq_T)\dots
\end{equation}
where the ellipses indicate terms that do not depend on
$q_T$. Therefore, using Eq.~(\ref{eq:besselproperty}) we find:
\begin{equation}
\begin{array}{l}
\displaystyle \int_{q_{T,\rm min}}^{q_{T,\rm
  max}}dq_T\left[\frac{d\sigma}{dQ dy dq_T} \right] \propto \int_0^\infty db\,
  \int_{q_{T,\rm min}}^{q_{T,\rm
  max}} dq_T\,   q_T J_0(bq_T)\dots= \\
\\
\displaystyle \int_0^\infty \frac{db}{b^2}\,
  \int_{bq_{T,\rm min}}^{bq_{T,\rm
  max}} dx\,   x J_0(x)\dots=\int_0^\infty \frac{db}{b}\left[q_{T,\rm
  max}J_1(bq_{T,\rm max}) - q_{T,\rm
  min}J_1(bq_{T,\rm min})\right]\dots\,.
\end{array}
\end{equation}
Therefore, defining:
\begin{equation}
K(q_T) \equiv \int dq_T\left[\frac{d\sigma}{dQ dy dq_T} \right]
\end{equation}
as the indefinite integral over $q_T$ of the cross section in
Eq.~(\ref{eq:crosssection}), we have that:
\begin{equation}\label{eq:primitive}
\int_{q_{T,\rm min}}^{q_{T,\rm
  max}}dq_T\left[\frac{d\sigma}{dQ dy dq_T} \right] = K(Q,y,q_{T,\rm max})
- K(Q,y,q_{T,\rm min})\,,
\end{equation}
with:
\begin{equation}
\begin{array}{l}
 \displaystyle K(Q,y,q_T) =
  \frac{16\pi\alpha^2q_T}{3N_c Q^3} H(Q,\mu) \\
\\
\displaystyle \times\sum_q C_q(Q)
  \frac12\int_0^\infty db\, J_1(bq_T) \overline{F}_q(x_1,b;\mu,\zeta) \overline{F}_{\bar{q}}(x_2,b;\mu,\zeta) f_{\rm NP}(x_1,b,\zeta)
  f_{\rm NP}(x_2,b,\zeta)\,,
\end{array}
\end{equation}
that can be computed using the Ogata quadrature as:
\begin{equation}
  K(Q,y,q_T) \simeq \sum_{n=1}^N w_n^{(1)} P\left(x_1,x_2,\frac{z_n^{(1)}}{q_T};\mu,\zeta\right) f_{\rm NP}\left(x_1,\frac{z_n^{(1)}}{q_T},\zeta\right) f_{\rm NP}\left(x_2,\frac{z_n^{(1)}}{q_T},\zeta\right)\,,
\end{equation}
where:
\begin{equation}
\begin{array}{rcl}
P(x_1,x_2,b;\mu,\zeta) &=&\displaystyle
  \frac1{b}S(x_1,x_2,b;\mu,\zeta) \\
\\
&=&\displaystyle\frac{16\pi\alpha^2}{3N_c Q^3}
    H(Q,\mu) \frac{1}{2}\sum_q C_q(Q) \left[\overline{F}_q\left(x_1,b_*(b);\mu,\zeta \right)\right] \left[\overline{F}_{\bar{q}}\left(x_2,b_*(b);\mu,\zeta \right)\right]\,,
\end{array}
\end{equation}
with $S$ defined in Eq.~(\ref{eq:pertfact}). The unscaled coordinates
$z_n^{(1)}$ and the weights $w_n^{(1)}$ can again be precomputed and
stored. Eq.~(\ref{eq:primitive}) reduces the integration in $q_T$ to a
calculation completely analogous to the unintegrated cross
section. This is particularly convenient because it avoids the
computation a numerical integration.

\subsection{On the position of the peak of the $q_T$ distribution}

It is interesting at this point to take a short detour to discuss the
postion of the peak on the distribution in $q_T$ of the cross section
in Eq.~(\ref{eq:crosssection}). The peak can be located by setting the
derivative in $q_T$ of the cross section equal to zero. To do so, we
use another property of Bessel's functions:
\begin{equation}
\frac{dJ_0(x)}{dx} = -J_1(x)\,.
\end{equation}
Using this relation, it is easy to see that:
\begin{equation}
\begin{array}{l}
\displaystyle 0 = \frac{d}{dq_T}  \left[\frac{d\sigma}{dQ dy dq_T}\right]
  =\\
\\
\displaystyle  \frac{16\pi\alpha^2}{3N_c Q^3} H(Q,\mu) \sum_q C_q(Q)
  \frac12\int_0^\infty db\,b \left[J_0(bq_T) -bq_TJ_1(bq_T)\right] 
\overline{F}_q(x_1,b_*(b);\mu,\zeta)
  \overline{F}_{\bar{q}}(x_2,b_*(b);\mu,\zeta)\\
\\
\displaystyle \times f_{\rm NP}(x_1,b,\zeta)
  f_{\rm NP}(x_2,b,\zeta)\,,
\end{array}
\end{equation}
that is equivalent to:
\begin{equation}
  \sum_q C_q(Q)
  \int_0^\infty db\,b \left[J_0(bq_T) -bq_TJ_1(bq_T)\right] 
  \overline{F}_q(x_1,b_*(b);\mu,\zeta) \overline{F}_{\bar{q}}(x_2,b_*(b);\mu,\zeta) f_{\rm NP}(x_1,b,\zeta)
  f_{\rm NP}(x_2,b,\zeta) = 0\,.
\end{equation}
The integral above can be solved numerically using the technique
discussed above and the value of $q_T$ that satisfies this equation
can be found.

\subsection{Integrating over $Q$ and $y$}

As a final step we need to perform the integrals over $Q$ and $y$
defined in Eq.~(\ref{eq:Intcrosssection}). To compute these integrals
we can only rely on numerical methods. Having reduced the integration
in $q_T$ to the difference of the two terms in the r.h.s. of
Eq.~(\ref{eq:primitive}), we can concentrate on integrating the
functions $K$ over $Q$ and $y$ for a fixed value of $q_T$:
\begin{equation}
\widetilde{K}(q_T)=\int_{Q_{\rm min}}^{Q_{\rm max}}dQ \int_{y_{\rm
    min}}^{y_{\rm max}}dy\,K(Q,y,q_T)\,,
\end{equation}
such that:
\begin{equation}\label{eq:Intcrosssection}
  \widetilde{\sigma} = \widetilde{K} (q_{T,\rm max})- \widetilde{K} (q_{T,\rm min})\,.
\end{equation}
To this purpose, it is convenient to make explicit the dependence of
$x_{1,2}$ on $Q$ and $y$ using Eq.~(\ref{eq:Bjorkenx12}). In addition,
for the sake of simplicity we will identify the scales $\mu$ and
$\sqrt{\zeta}$ with $Q$ (possible scale variations can be easily
reinstated at a later stage) and thus drop one of the arguments from
the TMD distributions $\overline{F}$ and from the hard factor $H$.
This yields:
\begin{equation}\label{eq:finalintegral}
\begin{array}{rcl}
\displaystyle  \widetilde{K}(q_T) &=& \displaystyle\int_0^\infty db\, J_1(bq_T)
  \frac{16\pi q_T}{3N_c} \int_{Q_{\rm min}}^{Q_{\rm max}}dQ\, 
  \frac{\alpha^2(Q)}{Q^3} H(Q) \sum_q C_q(Q)
  \frac12 \\
\\
&\times& \displaystyle 
                         \int_{e^{y_{\rm
    min}}}^{e^{y_{\rm max}}}d\xi\,\frac{1}{\xi} \,\overline{F}_q\left(\frac{Q}{\sqrt{s}}\xi,b_*(b);Q\right)
                         \overline{F}_{\bar{q}}\left(\frac{Q}{\sqrt{s}}\frac1{\xi},b_*(b);Q\right) \\
\\
&\times& \displaystyle f_{\rm NP}\left(\frac{Q}{\sqrt{s}}\xi,b;Q\right)
  f_{\rm NP}\left(\frac{Q}{\sqrt{s}}\frac1{\xi},b;Q\right)\,.
\end{array}
\end{equation}
Now we define a bidimensional grid in $\xi$, $\{\xi_\alpha\}$ with
$\alpha=0,\dots,N_\xi$, and in $Q$, $\{Q_\tau\}$ with
$\tau=0,\dots,N_Q$, each of which a set of interpolating functions $w$
associated. This allows us to interpolate the pair of functions
$f_{\rm NP}$ in Eq.~(\ref{eq:finalintegral}) for generic values of $\xi$
and $Q$ as:
\begin{equation}
f_{\rm NP}\left(\frac{Q}{\sqrt{s}}\xi,b;Q\right) f_{\rm NP}\left(\frac{Q}{\sqrt{s}}\frac1{\xi},b;Q\right) \simeq \sum_{\alpha=0}^{N_\xi}\sum_{\tau=0}^{N_Q}w_\alpha(\xi)w_\tau(Q) f_{\rm NP}\left(\frac{Q_\tau}{\sqrt{s}}\xi_\alpha,b;Q_\tau\right) f_{\rm NP}\left(\frac{Q_\tau}{\sqrt{s}}\frac1{\xi_\alpha},b;Q_\tau\right)\,.
\end{equation}
Plugging the equation above into Eq.~(\ref{eq:finalintegral}) we
obtain:
\begin{equation}
\begin{array}{rcl}
\displaystyle  \widetilde{K}(q_T) &\simeq& \displaystyle\int_0^\infty db\, J_1(bq_T)
  \sum_{\tau=0}^{N_Q}\sum_{\alpha=0}^{N_\xi}\Bigg[\frac{16\pi q_T}{3N_c} \int_{Q_{\rm min}}^{Q_{\rm max}}dQ\,w_\tau(Q)\, 
  \frac{\alpha^2(Q)}{Q^3} H(Q) \sum_q C_q(Q)
  \frac12 \\
\\
&\times& \displaystyle 
                         \int_{e^{y_{\rm
    min}}}^{e^{y_{\rm max}}}d\xi\,w_\alpha(\xi)\,\frac{1}{\xi} \,\overline{F}_q\left(\frac{Q}{\sqrt{s}}\xi,b_*(b);Q\right)
                         \overline{F}_{\bar{q}}\left(\frac{Q}{\sqrt{s}}\frac1{\xi},b_*(b);Q\right)\Bigg] \\
\\
&\times& \displaystyle f_{\rm NP}\left(\frac{Q_\tau}{\sqrt{s}}\xi_\alpha,b;Q_\tau\right) f_{\rm NP}\left(\frac{Q_\tau}{\sqrt{s}}\frac1{\xi_\alpha},b;Q_\tau\right)\,.
\end{array}
\end{equation}
Finally, the integration over $b$ can be performed using the Ogata
quadrature as discussed above, so that:
\begin{equation}
\begin{array}{rcl}
\displaystyle  \widetilde{K}(q_T) &\simeq& \displaystyle \sum_{n=1}^N
  \sum_{\tau=0}^{N_Q}\sum_{\alpha=0}^{N_\xi}\Bigg[w_n^{(1)}\frac{16\pi q_T}{3N_c} \int_{Q_{\rm min}}^{Q_{\rm max}}dQ\,w_\tau(Q)\, 
  \frac{\alpha^2(Q)}{Q^3} H(Q) \sum_q C_q(Q)
  \frac12 \\
\\
&\times& \displaystyle 
                         \int_{e^{y_{\rm
    min}}}^{e^{y_{\rm max}}}d\xi\,w_\alpha(\xi)\,\frac{1}{\xi} \,\overline{F}_q\left(\frac{Q}{\sqrt{s}}\xi,b_*\left(\frac{z_n}{q_T}\right);Q\right)
                         \overline{F}_{\bar{q}}\left(\frac{Q}{\sqrt{s}}\frac1{\xi},b_*\left(\frac{z_n}{q_T}\right);Q\right)\Bigg] \\
\\
&\times& \displaystyle f_{\rm NP}\left(\frac{Q_\tau}{\sqrt{s}}\xi_\alpha,\frac{z_n}{q_T};Q_\tau\right) f_{\rm NP}\left(\frac{Q_\tau}{\sqrt{s}}\frac1{\xi_\alpha},\frac{z_n}{q_T};Q_\tau\right)\,.
\end{array}
\end{equation}
In conclusion, if we define:
\begin{equation}
\begin{array}{rcl}
\displaystyle  W_{n\alpha\tau} & \equiv & \displaystyle w_n^{(1)}\frac{16\pi q_T}{3N_c} \int_{Q_{\rm min}}^{Q_{\rm max}}dQ\,w_\tau(Q)\, 
  \frac{\alpha^2(Q)}{Q^3} H(Q) \sum_q C_q(Q)
  \frac12 \\
\\
&\times& \displaystyle 
                         \int_{e^{y_{\rm
    min}}}^{e^{y_{\rm max}}}d\xi\,w_\alpha(\xi)\,\frac{1}{\xi} \,\overline{F}_q\left(\frac{Q}{\sqrt{s}}\xi,b_*\left(\frac{z_n}{q_T}\right);Q\right)
                         \overline{F}_{\bar{q}}\left(\frac{Q}{\sqrt{s}}\frac1{\xi},b_*\left(\frac{z_n}{q_T}\right);Q\right)\,,
\end{array}
\end{equation}
the quantity $\widetilde{K}(q_T)$ can be computed as:
\begin{equation}\label{eq:finalinterpolated}
\widetilde{K}(q_T) \simeq \sum_{n=1}^N
  \sum_{\tau=0}^{N_Q}\sum_{\alpha=0}^{N_\xi} W_{n\alpha\tau} f_{\rm NP}\left(\frac{Q_\tau}{\sqrt{s}}\xi_\alpha,\frac{z_n}{q_T};Q_\tau\right) f_{\rm NP}\left(\frac{Q_\tau}{\sqrt{s}}\frac1{\xi_\alpha},\frac{z_n}{q_T};Q_\tau\right)\,.
\end{equation}
The advantage of Eq.~(\ref{eq:finalinterpolated}) is that the weights
$W_{n\alpha\tau}$ can be precomputed once and for all and used to fit
the function $f_{\rm NP}$.

\subsection{Narrow-width approximation}

A possible alternative to the numerical integration in $Q$ when the
integration region includes the $Z$-peak region is the so-called
narrow-width approximation (NWA). In the NWA one assumes that the
width of the $Z$ boson, $\Gamma_Z$, is much smaller than its mass
$M_Z$. This way one can approximate the peaked behaviour of the
couplings $C_q(Q)$ around $Q=M_Z$ with a $\delta$-function,
\textit{i.e.}  $C_q(Q)\sim \delta(Q^2-M_Z^2)$. This way the
integration over $Q$ can be done analytically essentially setting
$Q=M_Z$ everywhere in the expression. The exact structure of the
electroweak couplings is the following:
\begin{equation}\label{eq:fullcoup}
C_q(Q) = e_q^2 - 2 e_q V_q V_e \chi_1(Q) + (V_e^2 + A_e^2)(V_q^2 + A_q^2)\chi_2(Q)\,,
\end{equation}
with:
\begin{equation}
\begin{array}{l}
\displaystyle \chi_1(Q) = \frac{1}{4 \sin^2\theta_W \cos^2\theta_W } \frac{Q^2 ( Q^2 -  M_Z^2 )}{ (Q^2 - M_Z^2)^2 + M_Z^2 \Gamma_Z^2} \,,\\
\displaystyle \chi_2(Q) = \frac{1}{16 \sin^4\theta_W\cos^4\theta_W} \frac{Q^4}{ (Q^2 - M_Z^2)^2 + M_Z^2 \Gamma_Z^2} \,.
\end{array}
\end{equation}
In the limit in which the width $\Gamma_Z$ is much smaller that the
$Z$ mass $M_Z$ ($\Gamma_Z/M_Z\rightarrow 0$), the leading contribution
to the coupling in Eq.~(\ref{eq:fullcoup}) comes from the region
$Q\simeq M_Z$ and is that proportional to $\chi_2$:
\begin{equation}\label{eq:partlead}
C_q(Q) \simeq (V_e^2 + A_e^2)(V_q^2 + A_q^2)\chi_2(Q)\,,\quad Q\simeq M_Z\,.
\end{equation}
In addition, in this limit one can show that:
\begin{equation}\label{eq:breitwigner}
\frac{1}{ (Q^2 - M_Z^2)^2 + M_Z^2 \Gamma_Z^2}\rightarrow
\frac{\pi}{M_Z\Gamma_Z}\delta(Q^2-M_Z^2) = \frac{\pi}{2M_Z^2\Gamma_Z}\delta(Q-M_Z)\,.
\end{equation}
Therefore, considering that:
\begin{equation}
\Gamma_Z = \frac{\alpha M_Z}{\sin^2\theta_W \cos^2\theta_W}\,,
\end{equation}
the electroweak couplings in the NWA have the following form:
\begin{equation}\label{eq:partlead}
C_q(Q) \simeq \frac{\pi M_Z (V_e^2 + A_e^2)(V_q^2 + A_q^2) }{32 \alpha
  \sin^2\theta_W\cos^2\theta_W} \delta(Q-M_Z)\,,
\end{equation}
such that the differential cross section in
Eq.~(\ref{eq:crosssection}) becomes:
\begin{equation}
\frac{d\sigma}{dQ dy dq_T} =
\frac{2 q_T\pi^2\alpha}{3N_c M_Z^2} H(M_Z,M_Z) \sum_q \frac{(V_e^2 + A_e^2)(V_q^2 + A_q^2) }{
  4\sin^2\theta_W\cos^2\theta_W} I_{q\bar{q}}(x_1,x_2,q_T;M_Z,M_Z^2) \delta(Q-M_Z)\,.
\end{equation}
Integrating the cross section over $Q$ under the condition that
$Q_{\rm min}<M_Z<Q_{\rm max}$ yields:
\begin{equation}
\frac{d\sigma}{dy dq_T} = \int_{Q_{\rm min}}^{Q_{\rm max}}dQ\,\frac{d\sigma}{dQ dy dq_T} =
\frac{2 q_T\pi^2\alpha}{3N_c M_Z^2} H(M_Z,M_Z) \sum_q \frac{(V_e^2 + A_e^2)(V_q^2 + A_q^2) }{
  4\sin^2\theta_W\cos^2\theta_W} I_{q\bar{q}}(x_1,x_2,q_T;M_Z,M_Z^2)\,.
\end{equation}
As a final step, one may want to let the $Z$ boson decay into
leptons. At leading order in the EW sector and assuming an equal decay
rate for electrons, muons, and taus, this can be done by multiplying
the cross section above by three times the branching ratio for the $Z$
decaying into any pair of leptons,
$\mbox{Br}(Z\rightarrow \ell^+\ell^-)$:
\begin{equation}
\frac{d\sigma}{dy dq_T} = 
\frac{2 q_T \mbox{Br}(Z\rightarrow \ell^+\ell^-)\pi^2\alpha}{N_c M_Z^2} H(M_Z,M_Z) \sum_q \frac{(V_e^2 + A_e^2)(V_q^2 + A_q^2) }{
  4\sin^2\theta_W\cos^2\theta_W} I_{q\bar{q}}(x_1,x_2,q_T;M_Z,M_Z^2)\,.
\end{equation}


\begin{thebibliography}{alp}

%\cite{Scimemi:2017etj}
\bibitem{Scimemi:2017etj}
  I.~Scimemi and A.~Vladimirov,
  %``Analysis of vector boson production within TMD factorization,''
  arXiv:1706.01473 [hep-ph].
  %%CITATION = ARXIV:1706.01473;%%
  %2 citations counted in INSPIRE as of 24 Oct 2017

%\cite{Collins:2011zzd}
\bibitem{Collins:2011zzd}
  J.~Collins,
  %``Foundations of perturbative QCD,''
  Camb.\ Monogr.\ Part.\ Phys.\ Nucl.\ Phys.\ Cosmol.\  {\bf 32} (2011) 1.
  %%CITATION = CMPCE,32,1;%%
  %327 citations counted in INSPIRE as of 21 Oct 2018

\bibitem{Ogata:quadrature}
  H.~Ogata,
  ``A Numerical Integration Formula Based on the Bessel Functions,''
  \texttt{http://www.kurims.kyoto-u.ac.jp/$\sim$okamoto/paper/Publ\_RIMS\_DE/41-4-40.pdf}

\end{thebibliography}

\end{document}
