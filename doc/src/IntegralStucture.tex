%% LyX 2.0.3 created this file.  For more info, see http://www.lyx.org/.
%% Do not edit unless you really know what you are doing.
\documentclass[twoside,english]{paper}
\usepackage{lmodern}
\renewcommand{\ttdefault}{lmodern}
\usepackage[T1]{fontenc}
\usepackage[latin9]{inputenc}
\usepackage[a4paper]{geometry}
\geometry{verbose,tmargin=3cm,bmargin=2.5cm,lmargin=2cm,rmargin=2cm}
\usepackage{color}
\usepackage{babel}
\usepackage{float}
\usepackage{bm}
\usepackage{amsthm}
\usepackage{amsmath}
\usepackage{amssymb}
\usepackage{graphicx}
\usepackage{esint}
\usepackage[unicode=true,pdfusetitle,
 bookmarks=true,bookmarksnumbered=false,bookmarksopen=false,
 breaklinks=false,pdfborder={0 0 0},backref=false,colorlinks=false]
 {hyperref}
\usepackage{breakurl}

\makeatletter

%%%%%%%%%%%%%%%%%%%%%%%%%%%%%% LyX specific LaTeX commands.
%% Because html converters don't know tabularnewline
\providecommand{\tabularnewline}{\\}

%%%%%%%%%%%%%%%%%%%%%%%%%%%%%% Textclass specific LaTeX commands.
\numberwithin{equation}{section}
\numberwithin{figure}{section}

%%%%%%%%%%%%%%%%%%%%%%%%%%%%%% User specified LaTeX commands.
\usepackage{babel}

\@ifundefined{showcaptionsetup}{}{%
 \PassOptionsToPackage{caption=false}{subfig}}
\usepackage{subfig}
\makeatother


\usepackage{listings}

\begin{document}

\section{Structure of the integrals}

The general structure of the quantities to be computed in {\tt APFEL}
is the combination of terms $I$ that have the form of Mellin
convolutions between an operator $O$ and distribution function $d$,
that is:
\begin{equation}\label{eq:ZMconv}
I(x) = x\int_0^1dz\int_0^1dy\, O(z)d(y)\delta(x-yz) = x\int_x^1\frac{dz}{z} O \left(\frac{x}{z}\right) d(z) = x\int_x^1\frac{dz}{z} O(z)d\left(\frac{x}{z}\right)\,.
\end{equation}
However, very often, typically in the presence of mass effects, the
integration phase space is modified and the convolution in
eq.~(\ref{eq:ZMconv}), limiting ourselves to the leftmost identity, is
generalised as:
\begin{equation}\label{eq:Genconv}
I(x,\eta) = x\int_{x/\eta}^1\frac{dz}{z} O(z,\eta)d\left(\frac{x}{\eta
    z}\right)\,.
\end{equation}
where $\eta\leq1$, with $\eta = 1$ reproducing
eq.~(\ref{eq:ZMconv}). However, for purposes that will become clear
later, we want to write the integral in eq.~(\ref{eq:Genconv}) in the
form of eq.~(\ref{eq:ZMconv}), that is in such a way that lower bound
of the integral is not the rescaled variable $x/\eta$ but the physical
Bjorken $x$. To do so, one needs to perform the change of variable
$y =\eta z$, so that:
\begin{equation}\label{eq:Genconv1}
  I(x,\eta)= \int_x^\eta dy\,
  O\left(\frac{y}{\eta},\eta\right)\,\frac{x}{y}d\left(\frac{x}{y}\right)\,.
\end{equation}

In order to precompute the expensive part of the integral in
eq.~(\ref{eq:Genconv1}), we use the standard interpolation formula to
the distribution $d$:
\begin{equation}\label{eq:Interpolation}
\frac{x}{y} d\left(\frac{x}{y}\right) = \sum_{\alpha=0}^{N_x} x_\alpha
d(x_\alpha) w_{\alpha}^{(k)}\left(\frac{x}{y}\right)\,,
\end{equation}
where $\alpha$ runs over the node of a give grid in $x$ space and the
weights $w_{\alpha}$ are typically polynomials of degree $k$
($i.e.$ Laplace interpolants). Now we use eq.~(\ref{eq:Interpolation})
in eq.~(\ref{eq:Genconv1}) and at the same time we limit the
computation of the integral $I$ to the point $x_\beta$ of the grid
used in eq.~(\ref{eq:Interpolation}). This way we get:
\begin{equation}\label{eq:Genconv1}
  I(x_\beta,\eta)=  \sum_{\alpha=0}^{N_x} \overline{d}_\alpha\int_{x_\beta}^\eta dy\,
  O\left(\frac{y}{\eta},\eta\right)\,w_{\alpha}^{(k)}\left(\frac{x_\beta}{y}\right)\,.
\end{equation}
where we have defined $\overline{d}_\alpha = x_\alpha d(x_\alpha)$.
In the particular case of the Laplace interpolants (in the {\tt APFEL}
procedure), one can show that:
\begin{equation}
w_{\alpha}^{(k)}\left(\frac{x_\beta}{y}\right)\neq 0\quad\mbox{for}\quad c < y < d\,,
\end{equation}
with:
\begin{equation}
  c =
  \mbox{max}(x_\beta,x_\beta/x_{\alpha+1}) \quad\mbox{and}\quad d =
  \mbox{min}(\eta,x_\beta/x_{\alpha-k}) \,,
\end{equation}
and thus eq.~(\ref{eq:Genconv1}) can be adjusted as:
\begin{equation}\label{eq:Genconv2}
  I(x_\beta,\eta)=  \sum_{\alpha=0}^{N_x} \overline{d}_\alpha\int_c^d dy\,
  O\left(\frac{y}{\eta},\eta\right)\,w_{\alpha}^{(k)}\left(\frac{x_\beta}{y}\right)\,.
\end{equation}
Finally, we change back the integration variable $z=y/\eta$ so that
eq.~(\ref{eq:Genconv2}) becomes:
\begin{equation}\label{eq:Genconv3}
  I(x_\beta,\eta)=  \sum_{\alpha=0}^{N_x} \overline{d}_\alpha \underbrace{\left[\eta\int_{c/\eta}^{d/\eta} dz\,
  O\left(z,\eta\right)\,w_{\alpha}^{(k)}\left(\frac{x_\beta}{\eta z}\right)\right]}_{\Gamma_{\alpha\beta}}\,.
\end{equation}
The quantity in squared brackets is the interesting bit.

The expressions for the operator $O$ that we have to deal with have
this general form:
\begin{equation}\label{eq:CoeffFuncs}
O(z,\eta) = R(z,\eta)+\sum_{i=0}^{n}\left[\frac{\ln^i(1-z)}{1-z}\right]_+ S^{(i)}(z,\eta)+ L(\eta)\delta(1-z)\,,
\end{equation}
where $R$ and $S^{(i)}$ is a regular function in $z=1$, that is:
\begin{equation}
R(1,\eta) =  \lim_{z\rightarrow 1} R(z,\eta) = K(\eta)\quad\mbox{and}\quad S^{(i)}(1,\eta) =  \lim_{z\rightarrow 1} S^{(i)}(z,\eta) = J^{(i)}(\eta)\,,
\end{equation}
being $K$, $J^{(i)}$, and $L$ a finite function of $\eta$. Notice that
the sum over $i$ in eq.~(\ref{eq:CoeffFuncs}) is between zero and $n$
where $n$ is typically not bigger than two as it depends on the
perturbative order of the expressions. Plugging
eq.~(\ref{eq:CoeffFuncs}) into the definition of
$\Gamma_{\alpha\beta}$ in eq.~(\ref{eq:Genconv3}) and making use of
the definition of plus-prescription, we obtain:
\begin{equation}\label{eq:FinalExpression}
\begin{array}{rcl}
\Gamma_{\beta\alpha} &=& \eta\displaystyle \int_{c/\eta}^{d/\eta} dz\left\{R\left(z,\eta\right)w_{\alpha}^{(k)}\left(\frac{x_\beta}{\eta z}\right)+\sum_{i=0}^{n}\frac{\ln^i(1-z)}{1-z}\left[S^{(i)}\left(z,\eta\right)w_{\alpha}^{(k)}\left(\frac{x_\beta}{\eta z}\right)-S^{(i)}(1,\eta)w_{\alpha}^{(k)}\left(\frac{x_\beta}{\eta}\right)\right]\right\}\\
\\
&+&\displaystyle \eta\left[\sum_{i=0}^{n} \frac1{(i+1)!}S^{(i)}(1,\eta)\ln^{i+1}\left(1-\frac{c}{\eta}\right)+L(\eta) \right]w_{\alpha}^{(k)}\left(\frac{x_\beta}{\eta}\right)\,.
\end{array}
\end{equation}

Eq.~(\ref{eq:FinalExpression}) expresses the full complexity of the
task. However, there are a few remarks that apply is some particular
cases and that reduce the complexity. In the case of the
($\overline{\mbox{MS}}$) splitting functions, there are two main
simplications: the first is that $\eta=1$, the second is that $n=0$ in
the sums. Considering that:
\begin{equation}
w_\alpha^{(k)}(x_\beta) = \delta_{\alpha\beta}\,,
\end{equation}
and that the expressions can be manipulated in such a way that the
coefficient of the plus-prescripted term $S$ is a constant, we have that:
\begin{equation}\label{eq:SplittingFunctions}
\Gamma_{\beta\alpha} = \int_{c}^{d} dz\left\{R\left(z\right)w_{\alpha}^{(k)}\left(\frac{x_\beta}{z}\right)+\frac{S}{1-z}\left[w_{\alpha}^{(k)}\left(\frac{x_\beta}{z}\right)-\delta_{\alpha\beta}\right]\right\}+\displaystyle \left[S\ln\left(1-c\right)+L \right]\delta_{\alpha\beta}\,.
\end{equation}

The same kind of simplifications apply to the case of the Zero-Mass
(ZM) coefficient functions with the only exception that the sum over
$i$ extends to $n$'s larger than zero. In particular, since we will
use exact expressions only up to $\mathcal{O}(\alpha_s)$, $i.e.$ NLO,
we have $n=1$ so that:
\begin{equation}\label{eq:ZMCoeffFunctions}
\begin{array}{rcl}
  \Gamma_{\beta\alpha} &=&\displaystyle \int_{c}^{d}
                           dz\left\{R\left(z\right)w_{\alpha}^{(k)}\left(\frac{x_\beta}{z}\right)
  + \frac{S^{(0)}+S^{(1)}\ln(1-z)}{1-z}\left[w_{\alpha}^{(k)}\left(\frac{x_\beta}{z}\right)-\delta_{\alpha\beta}\right]\right\}\\
\\
&+&\displaystyle
    \left[S^{(0)}\ln\left(1-c\right)+\frac12
    S^{(1)}\ln^2\left(1-c\right)+L \right]\delta_{\alpha\beta}\,.
\end{array}
\end{equation}

As far as the massive coefficient functions up to
$\mathcal{O}(\alpha_s^2)$ things can be more complicated and we will
discuss later. In the next section we will consider the case of
hadronic observables in the ZM scheme showing how to pre-compute the
integral for the double-differential distributions in Drell-Yan
production.
















\end{document}
